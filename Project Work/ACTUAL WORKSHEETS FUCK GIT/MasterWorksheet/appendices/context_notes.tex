\chapter{Notes Taken During the Observation}\label{ch:surgery_obs}
\begin{itemize}
	\item If you are not sterile, you have to stand at least one meter from sterile objects.
	\item The tables covered in green paper blankets are sterile.
	\item When preparing tools and unpacking, two nurses works together. One non-sterile nurse unpacks while the other, sterile, nurse grabs and places the tools on the sterile table.
	\item The sterile nurse sterilises the robot.
	\item When the patient enters the room he is firstly laid down on the operating table and then prepared for surgery.
	\item The stereoscopic camera is wrapped separately from the robot and other tools.
	\item When all sterile tools are placed, the sterile table is covered.
	\item The camera is calibrated by the sterile nurse. This is done using different kinds of end pieces and rotating them around scopes.
	\item The arms of the robot must be placed in a specific order. This is to avoid any kind of collision of the arms. Furthermore, the placement of the arms is as important as the order. This is done before the robot is docked.
	\item Before the robot is docked, the nurses taking part in the operation are sterilising.
	\item When the robot is docked the arms are once again placed, this time around the ports inserted on the patient.
	\item A time out is taken before any cutting securing everyone and everything is ready and in place.
	\item The first cut done on the patient is to expand his stomach using air, easing the operation as this will yield more space.
	\item Before docking the camera on the robot it is used hand held to insert the other instruments and afterwards docked on the dedicated robot arm.
	\item When the camera is docked the robot arms are placed as far apart as possible. This is done to avoid collision.
	\item Each arm has three buttons enabling arm movement in two different ways.
	\item Cleaning of the optics is done several times during the operation. it may even be changed for another if it is too dirty.
	\item Communication during operation is somewhat an issue as the original speaker system made for this kind of surgery is broken. Instead, a small speaker and bad microphone is used.
\end{itemize}