\chapter*{Training Session}
\section*{Purpose}
The purpose of the interview is to gain understanding of how nurses and surgeons are training to operate a da Vinci robot and perform (RAMIS). 

\section*{Method}
We performed a semi-structured interview with one of the nurses who train students, Jane Petersson and a surgeon, Johan Poulsen, at MUIC.
 
\section*{Results}
The training of the students consists of boht a theoretical course and a practical session. During the theoretical course, students learn about the equipment and how to operate it. In the practical session, a group of four students perform an operation on a live pig. During the session, students change roles and perform a variety of tasks. An instructor is giving instructions about their  have to do. The purpose of the training is for the students to get familiar with the robot and perform practical tasks while focusing on communication between team members. Before the surgery, the instructor usually sabotages parts of the procedure to simulate unexpected situation for the students.
The team consists of a surgeon, first assistant and nurses. The surgeon operates the robot while the first assistant changes tools and assists with tasks that require work inside the patient's body. The nurses prepare tools for the first assistant during the procedure and sanitise the robot before the surgery.
 

\section*{Conclusion}
The interview has given an insight into the training situation of the nurses working around and with the robot.