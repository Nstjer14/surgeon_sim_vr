\section{Discussion}

The concept of introducing VR training in established RAMIS training sessions seems feasible, however with some caveats. Currently, the system is too basic to warrant implementation at MIUC according to the head surgeon, but could serve other purposes such as introducing medical students to RAMIS. Experts stated that the system would require a high level of realism to accurately show the procedure and be useful in RAMIS training. This means that key considerations for future implementations would be the introduction of multiple control points on the da Vinci robot's arms, followed by a comparison of inverse kinematic methods to simulate their movement. This would also require a fully modelled and rigged robot with all the joints of the real one. The robot's interaction with important tools such as surgery ports will be similarly important, for example by ``clicking" into place. 

As this study has been exploratory, it risks being positively valued without real life feasibility. To assess the actual impact of a system such as this, we suggest a longitudinal study using a fully implemented system. Installing this concept at institutes such as the minimally invasive education centre would require knowledge of system usability as well as impact on training performance, and time- and cost savings. Expert statements have been very valuable, and more RAMIS instructors would be necessary to define which functions should be available to instructors and trainees, such as scene change and resetting. 

Since the knowledge of roles and their individual tasks is rather important during real surgeries, a single user option is a possibility. This could enable trainees to learn the objectives of all the staff. This would require a series of pre-set scenarios with clear goals and indications of correctly performed tasks. Being able to control even more factors, such as patient fever, would allow for more detailed training of both nurses and surgeons. This concept could be extended to include multiple scenarios and allow instructors to control scenarios. Having multiple scenarios available without instructor interference could improve the utility of the system, for example by allowing students to learn how to dock the robot for specific surgeries at their discretion, but would similarly require visible goals and indications of successful tasks.

This study has paved the way for a novel way to train RAMIS nurses and surgeons. Currently, it seems team training in VR is a viable alternative to regular training with the potential to save both money and time for both trainees and instructors. Confirming this has enabled a plethora of exploratory studies to determine whether training specific RAMIS scenarios such as docking and undocking the robot, emergency handling, and even full surgery simulations, are similarly viable.

%less joints, use other IK solver
%This could allow full surgery team training. As noted from the context study, Jane Petersson sabotages the robot setup during team training in order to train the nurses' and doctors' communication and emergency handling skills. Being able to control even more factors in virtual reality, such as patient fever, would allow for more detailed training of both nurses and surgeons. This concept could also be extended to include scenarios and scenario control. Having multiple scenarios and tasks to follow would improve the utility of the system, for example by allowing students to learn how to dock the robot for specific surgeries.

%The current system was made to test the viability of a fully implemented training simulation. The training sessions at MIUC focus on certain key points such as communication, precision, and hands-on experience. However, implementing a system that is capable of simulating entire training scenarios where all these competencies are in focus is not possible with the available resources. As such, emergency undocking was chosen as it was deemed important for working with the da Vinci Surgical System as to not increase trauma risks for the patient.

%The usability test conducted with the system indicated that it performed above average with a mean score of $70.3$. This puts it in the $69^{th}$ percentile. According to Bangor et al, if the results are higher than $52.01$ the interface is considered OK whereas $72.75$ and above is good \citep{bangor_empirical_2008}. The average score is 68 \citep{sauro_sustisfied?_nodate}. The expert review revealed a promising future for the simulation of surgical robotics training, but established that the current system was insufficient due to its lack of scenarios. This indicates that the concept is viable, but requires further work to warrant implementation at MIUC and similar institutes.

%The system was made for multiplayer use, as communication and teamwork are key factors in successful surgeries. As observed during team training, it is important for trainees to know the tasks and roles of the rest of the medical personnel. Another important factor in team training is to get familiar with the robot and increase time spent with it. As such, a single-player version of the simulation could be developed to allow for using the system without any other participants. Furthermore, having a single-player simulation would also allow the user to perform all the tasks required for surgery and thus learn the tasks of every member of the surgical team. Additionally, improving the accuracy of the virtual robot's arms' physics would allow for more demanding tasks and scenarios to train in VR. These could include trainee surgeons to extend team training to full-scale operations.

%In order to validate the usability of the system it is important to use participants already familiar with the HTC Vive. As learned from the user studies some users would learn how to use the VR system rather than learn the actual training from the simulation.