\chapter*{Second Interview}

\section*{Purpose}
The purpose of the interview was to understand how nurses and surgeons train to operate a da Vinci robot and perform (RAMIS). 

\section*{Method}
We performed a semi-structured interview with one of the nurses who train students, Jane Petersson, at MUIC. 

\section*{Results}
One of the most important things during surgery and training is the emergency handling, however this requires extensive knowledge of the robot and tools. Another very important task is the placement of the robot arms because collisions during surgery can be very dangerous. Sometimes the surgeons adjust the arms manually.

In general, the training is for hands on experience with the robot and individual instruments, in preparation for a real surgery.

The team training consists of:
\begin{itemize}
  \item First assistant nurse - Sterile, assists the surgeon inside the patient's body.
  \item Scrub nurse - Sterile, prepare unpacked tools for the first assistant nurse.
  \item Circulating nurse - Not sterile, unpacks tools, and can go outside the sterile area.
\end{itemize}  

During training, on the first day Jane Petersson starts by talking and showing how tasks are done, students observe. On the second day, students are expected to do most of the tasks themselves. At the end of the second day, Jane Petersson sabotages some of the equipment and the students have to find and fix it. Any mistakes done by the students are used for learning purpose. At the beginning of the training, Jane Petersson makes most of the decisions, however later the surgeon is expected to take the lead.



\section*{Conclusion}