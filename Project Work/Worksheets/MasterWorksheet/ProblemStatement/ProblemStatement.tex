\chapter*{Problem Statement}

\section*{Motivation}
Robot assisted minimally invasive surgery (RAMIS) is used for many medical procedures. It allows for more precise surgery and lowers the risk of injury while reducing recovery time for patients [History of minimally invasive surgery]. Traditionally, the surgical performance has been measured in terms of surgeons' visuo-motor skills and dexterity, however a shift in the conceptualization of competencies is emerging in new literature (Safer Surgery book p 83). One of these competencies is teamwork in the operating theatre (source Hayley et al). A study concluded that 80\% of incidents in anaesthesia related errors were preventable (Reported significant observations during anaesthesia: a prospective analysis over an 18-month period). Of these, 75\% of these were due to human error. 

Lingard et al studied communication in the operating theatre and found that 36\% of errors affected the teamwork negatively (Communication failures in the operating room: an observational classification of recurrent types and effects). From this study, five key points of communication in the operating theatre were found: time, resources, roles, safety and sterility, and situation control.


To counter human errors, and to reduce critical events, simulation based training has become increasingly popular over the years (The simulated operating theatre: comprehensive training for surgical teams). This applies to multiple professional and high risk contexts such as aviation and surgery. As the goal of simulated training is to involve all members of the surgical staff and to enhance their performance, R. Aggarwal suggests that the method can also be applied to other procedures such as robot surgery. As robot surgery is also a high risk context, virtual simulations of high risk procedures is also beneficial. 
This paper will explore the possibilities of simulating a medical emergency during surgery training in Virtual Reality (VR) and document the effects on teamwork and performance. 

\subsection*{Problem Statement}

Based on the information above and focusing on the impact of recent Virtual Reality technology, this paper aims to answer the following problem statement:\\

\begin{center}
\emph{\textit{"How can virtual reality be utilized to simulate an emergency situation during RAMIS training?"}}
\end{center}


\subsection*{Success Criteria}

Up for discussion. What do you think Martin? 

Suggestions: 

- have a working prototype
- have a way of measuring teamwork
- have a way of measuring performance
- have a procedure for testing the above and plan statistical test
