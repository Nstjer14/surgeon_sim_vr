\chapter{Problem Statement}

\section{Motivation}

Robot assisted minimally invasive surgery (RAMIS) is used for many medical procedures. It allows for more precise surgery and lowers the risk of injury while reducing recovery time for patients \citep{radojcic_[history_2009}. Traditionally, the surgical performance has been measured in terms of surgeons' visuo-motor skills and dexterity, however a shift in the conceptualization of competencies is emerging in new literature \citep{flin_safer_2009}[p. 83]. One of these competencies is teamwork in the operating theatre \citep{salas_does_2008}. 

Lingard et al studied communication in the operating theatre and found that 36\% of errors affected the teamwork negatively \citep{lingard_communication_2004}. From this study, five key points of communication in the operating theatre were found: time, resources, roles, safety and sterility, and situation control.


To counter human errors, and to reduce critical events, simulation based training has become increasingly popular over the years \citep{aggarwal_simulated_2004}. This applies to multiple professional and high risk contexts such as aviation and surgery. As the goal of simulated training is to involve all members of the surgical staff and to enhance their performance, R. Aggarwal suggests that the method can also be applied to other procedures such as robot surgery. As robot surgery is also a high risk context, virtual simulations of high risk procedures is also beneficial. 
This project will explore the possibilities of simulating surgery training in virtual reality (VR) and assess the viability in context. 


%------------------------------------------------

\subsection{Problem Statement}

Based on the information above and focusing on the impact of recent virtual reality technology, this project aims to answer the following problem statement:\\

\begin{center}
\emph{\textit{"Is virtual reality a viable addition to RAMIS team training?"}}
\end{center}


