\chapter*{Final System}
\section*{Description} 
The purpose of the project is to develop a simulation environment for training nurses in participating in RAMIS. The simulation needs to allow trainees to practice different tasks and scenarios that might occur during real surgery or be part of physical training.
The simulation should allow for three to four trainees and an instructor to be simultaneously in the same virtual operating room. In the room, trainees have access to all the tools and equipment that would be available to them in a physical training. At this point we are simulating only emergency undocking since it is one of the most important asks during procedure according to Jane Petersson.
The simulation allows trainees to walk to move on short distance or teleport for long distance movement. Interactable objects in the simulation, such as tools and the robot, can be grabbed to carry and use. Tools are two types, hand tools and robot equipment. The hand tools are used mainly by the first assisting nurse. The robot equipment includes the camera and tools that can be attached to the arms. Trainees can interact with the robot by moving its arms to achieve optimal placement, which is required for proper functioning of the robot. Additionally, voice over IP (VOIP) service is implemented to support communication between the trainees. 





\section*{Conclusion}
