\chapter{Interview Script Including Notes}\label{sec:interview_script}
We firstly introduce Jane and Johan to the simulation. They will get to watch one from the group “play around”, while being explained how the controls work etc. 
The robot will be in un docked position, but the patient is still on the table. The 4 ports, camera and 3 tools are presented on the tool table, they can insert these and “play around”. 
Afterwards they will be able to try the simulation themselves and evaluate the scene.

\section{Simulation Feasibility}
\textit{Which aspects of the simulation works well?}
\begin{itemize}
	\item The basic functionality is working well.
	\item The squares which was used to control the robot may need an indication of the orientation
	\item Nice that the instruments could be socketed
	\item The room itself worked well
\end{itemize}

\textit{Which aspects of the simulation does not work well?}
\begin{itemize}
	\item The cart needs to be moved closer to the patient.
	\item The arms didn’t work properly, as they weren’t moving in the same as the real robot did.
	\item The instruments should be able to move instead of just pointing down. 
\end{itemize}

\textit{Is anything crucial missing from the simulation?}
\begin{itemize}
	\item More realism in general. 
	\item You should be able to see the other “players” with each their dedicated roles.
	\item Working together.
	\item Being able to go from the nurse position and docking the robot to sitting at the console and operate the robot. – A full simulation of the surgery.
	\item Even expanding with anaesthetics and information hereof.
	\item Implementing disaster/accidents as well, which you would need to adept to.
\end{itemize}

\textit{Is anything in the simulation redundant/superfluous?}
\begin{itemize}
	\item No
\end{itemize}

\textit{How would you describe the realism of the simulation? How did it affect the experience?}
\begin{itemize}
	\item Nice and spacious.
	\item Needs more realism to be considered a useful simulation
\end{itemize}

\textit{What details did you find missing/lacking in the scene?}
\begin{itemize}
	\item The console is missing from the simulation
	\item Ports in the patient, to give an end goal for the. Maybe attach the arms to the ports and make insertion of the tools in the ports.
	\item When the arms are docked they should still be movable, but around the port socket of course.
\end{itemize}

\section{Further Work}
\textit{What is the next step for us to implement?}
\begin{itemize}
	\item 5 ports and making the arms’ endpoints socket to the ports. 
	\item Results of too much force
	\item It is too basic, but the idea has great potential.
	\item Refinement and realism are key points.
\end{itemize}

\textit{What would you use this kind of simulation for? Opportunities, purposes, direction of development.}
\begin{itemize}
	\item As of right now, it may be able to give a basic idea to a completely new user. But it’s not detailed enough. People will not be able to train with this simulation. It has a lot of simulation.
	\item It’s too basic right, but with further development, it can yield enormous possibilities. 
\end{itemize}

\textit{Could it be used for showcasing?}
\begin{itemize}
	\item As before, it should be refined and more realistic, but it is a possibility to use it to showcase what is going on, how the team work together and what is going on.
\end{itemize}

\textit{Do the controls suit your needs? How?}
\begin{itemize}
	\item The two control options are good.
	\item A reset button for a “mentor”/admin/supervisor. 
	\begin{itemize}
		\item A laser pointer which should be visible for everyone. 
		\item Regular controls as well.
	\end{itemize}
	\item Admin controls to switch pre-set scenarios and the different kinds of operations/setups.
\end{itemize}

\textit{Which scenarios are important as well?}
\begin{itemize}
	\item Role designation
\end{itemize}

\textit{What would be nice to have later on?}
\begin{itemize}
	\item A complete simulation of a surgery where you’re able to do everything which is done in real life.
	\item Tactile feedback would be nice – but you use your eyes and compensate as you know the visual signals.
	\begin{itemize}
		\item We could use the vibration in the controllers
	\end{itemize}
	
\end{itemize}