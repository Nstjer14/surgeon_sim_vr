%%%%%%%%%%%%%%%%%%%%%%%%%%%%%%%%%%%%%%%%%
% Short Sectioned Assignment
% LaTeX Template
% Version 1.0 (5/5/12)
%
% This template has been downloaded from:
% http://www.LaTeXTemplates.com
%
% Original author:
% Frits Wenneker (http://www.howtotex.com)
%
% License:
% CC BY-NC-SA 3.0 (http://creativecommons.org/licenses/by-nc-sa/3.0/)
%
%%%%%%%%%%%%%%%%%%%%%%%%%%%%%%%%%%%%%%%%%

%----------------------------------------------------------------------------------------
%	PACKAGES AND OTHER DOCUMENT CONFIGURATIONS
%----------------------------------------------------------------------------------------

\documentclass[paper=a4, fontsize=11pt]{scrartcl} % A4 paper and 11pt font size

\usepackage[T1]{fontenc} % Use 8-bit encoding that has 256 glyphs
\usepackage{fourier} % Use the Adobe Utopia font for the document - comment this line to return to the LaTeX default
\usepackage[english]{babel} % English language/hyphenation
\usepackage{amsmath,amsfonts,amsthm} % Math packages

\usepackage{lipsum} % Used for inserting dummy 'Lorem ipsum' text into the template

\usepackage{sectsty} % Allows customizing section commands
\allsectionsfont{\centering \normalfont\scshape} % Make all sections centered, the default font and small caps

\usepackage{graphicx} %Figures

\usepackage{fancyhdr} % Custom headers and footers
\pagestyle{fancyplain} % Makes all pages in the document conform to the custom headers and footers
\fancyhead{} % No page header - if you want one, create it in the same way as the footers below
\fancyfoot[L]{} % Empty left footer
\fancyfoot[C]{} % Empty center footer
\fancyfoot[R]{\thepage} % Page numbering for right footer
\renewcommand{\headrulewidth}{0pt} % Remove header underlines
\renewcommand{\footrulewidth}{0pt} % Remove footer underlines
\setlength{\headheight}{13.6pt} % Customize the height of the header

\numberwithin{equation}{section} % Number equations within sections (i.e. 1.1, 1.2, 2.1, 2.2 instead of 1, 2, 3, 4)
\numberwithin{figure}{section} % Number figures within sections (i.e. 1.1, 1.2, 2.1, 2.2 instead of 1, 2, 3, 4)
\numberwithin{table}{section} % Number tables within sections (i.e. 1.1, 1.2, 2.1, 2.2 instead of 1, 2, 3, 4)

\setlength\parindent{0pt} % Removes all indentation from paragraphs - comment this line for an assignment with lots of text

%----------------------------------------------------------------------------------------
%	TITLE SECTION
%----------------------------------------------------------------------------------------

\newcommand{\horrule}[1]{\rule{\linewidth}{#1}} % Create horizontal rule command with 1 argument of height

\title{	
\normalfont \normalsize 
\textsc{Aalborg University} \\ [25pt] % Your university, school and/or department name(s)
\horrule{0.5pt} \\[0.4cm] % Thin top horizontal rule
\huge Worksheet - Rules \\ % The assignment title
\horrule{2pt} \\[0.5cm] % Thick bottom horizontal rule
}

\date{\normalsize\today} % Today's date or a custom date

\begin{document}

\maketitle % Print the title

%----------------------------------------------------------------------------------------
%	PROBLEM 1
%----------------------------------------------------------------------------------------

\section{Purpose}

The purpose was to test the system used in the expert review, including hardware load and frame times. This was done to ensure a "smooth" (pls fix) test and to determine whether potential issues could be attributed to physical limitations.

\subsection{Methods}

The test was performed at AAU's Audio-Visual Arena using the computers described in \autoref{tab:specs}. 

\begin{table}
\centering
\begin{tabularx}{0.48\textwidth}{X X X X}
\toprule
                     & \textit{Computer A} & \textit{Computer B} & \textit{Computer C} \\ \midrule \rowcolor{lightGrey}
\textbf{CPU}         & Intel Core i7-6700K & Intel Core i7-4770  & Intel Core i7-6700  \\

\textbf{GPU}         & Nvidia GTX 1080     & Nvidia GTX 980      & Nvidia GTX 980    \\  \rowcolor{lightGrey}

\textbf{RAM} 		 & 32GB                & 16GB                & 16GB                 \\ \toprule
\end{tabularx}
\caption{Specifications of the computers used}
\label{tab:specs}
\end{table}

Two players and two observers were present. The players moved around in an introduction scene and in the final scene used in the system review. The observers recorded data using MSI Afterburner and Fraps to measure CPU and GPU load as well as framerate.

\subsection{Results}
The users in the scene should be designated a specific role which entails different tasks depending on the amount of users in the current session. While completing the individual tasks is important, the primary focus of this exercise is to complete the overall task of saving the patient as one unit. This means that communication and cooperation is very important to the success of this goal.

\subsection{Conclusion}
The rules for the equipment is based on the observations and interview with Jane. These rules entails how the equipment should be used and handled by the users. As most of the equipment is only used for one specific use, it important that the users are aware of that specific use.

%----------------------------------------------------------------------------------------

\end{document}