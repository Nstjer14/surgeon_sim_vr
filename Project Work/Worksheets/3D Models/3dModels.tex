\documentclass[paper=a4, fontsize=11pt]{scrartcl} % A4 paper and 11pt font size

\usepackage[T1]{fontenc} % Use 8-bit encoding that has 256 glyphs
\usepackage{fourier} % Use the Adobe Utopia font for the document - comment this line to return to the LaTeX default
\usepackage[english]{babel} % English language/hyphenation
\usepackage{amsmath,amsfonts,amsthm} % Math packages
\usepackage{graphicx}
\usepackage{float}

\usepackage{lipsum} % Used for inserting dummy 'Lorem ipsum' text into the template

\usepackage{sectsty} % Allows customizing section commands
\allsectionsfont{\centering \normalfont\scshape} % Make all sections centered, the default font and small caps

\usepackage{fancyhdr} % Custom headers and footers
\pagestyle{fancyplain} % Makes all pages in the document conform to the custom headers and footers
\fancyhead{} % No page header - if you want one, create it in the same way as the footers below
\fancyfoot[L]{} % Empty left footer
\fancyfoot[C]{} % Empty center footer
\fancyfoot[R]{\thepage} % Page numbering for right footer
\renewcommand{\headrulewidth}{0pt} % Remove header underlines
\renewcommand{\footrulewidth}{0pt} % Remove footer underlines
\setlength{\headheight}{13.6pt} % Customize the height of the header

\numberwithin{equation}{section} % Number equations within sections (i.e. 1.1, 1.2, 2.1, 2.2 instead of 1, 2, 3, 4)
\numberwithin{figure}{section} % Number figures within sections (i.e. 1.1, 1.2, 2.1, 2.2 instead of 1, 2, 3, 4)
\numberwithin{table}{section} % Number tables within sections (i.e. 1.1, 1.2, 2.1, 2.2 instead of 1, 2, 3, 4)

\setlength\parindent{0pt} % Removes all indentation from paragraphs - comment this line for an assignment with lots of text

%----------------------------------------------------------------------------------------
%	TITLE SECTION
%----------------------------------------------------------------------------------------

\newcommand{\horrule}[1]{\rule{\linewidth}{#1}} % Create horizontal rule command with 1 argument of height

\title{	
\normalfont \normalsize 
\textsc{Aalborg University, School of Information and Communication Technology} \\ [25pt] % Your university, school and/or department name(s)
\horrule{0.5pt} \\[0.4cm] % Thin top horizontal rule
\huge 3D Models \\ % The assignment title
\horrule{2pt} \\[0.5cm] % Thick bottom horizontal rule
}

\date{\normalsize\today} % Today's date or a custom date

\begin{document}

\maketitle % Print the title
\section{Prioritizing the Implementation}

A Need / Want / Nice to have list was created to scope the project and to prioritize essential features for the evaluation. The needs in this list constitutes to the minimum implementation requirements and contains the most important features and functionalities to test the system qualitatively and quantitatively. 

\textbf{Need}
\textbf{Want}
\textbf{Nice to have}
 

\section{Tools Used}
Various programs were used to recreate the operating theatre in 3D. The scene layout was created in Unreal Engine 4 from the physical model described in Section \ref{•}.Unreal Engine 4 was chosen over Unity as implementation of multi-player in virtual reality proved easier. 
The 3D models in the scene was created using Audtodesk Maya and 3DS Max. These programs support box modelling which was used to shape all the objects. 
Materials were added later when each of the objects were imported to Unreal Engine 4. They were kept simple to improve performance and visual simplicity.

\begin{figure}[H]
\centering
\includegraphics[width=\textwidth]{tools.png}
\caption{Logos of the tools used}
\end{figure}


%------------------------------------------------

\section{da Vinci Robot}

How the da Vinci robot was made and how the IK handles was implemented

\begin{figure}[H]
\centering
\includegraphics[width=\textwidth]{plz.png}
\caption{The da Vinci Robot made in 3Ds Max}
\end{figure}

\section{Operating Theatre in Virtual Reality}
Insert screenshots and write about the final room in the scene. 

\section{Problems Encountered}

IK handles in UNREAL ENGINE.


%----------------------------------------------------------------------------------------

\end{document}