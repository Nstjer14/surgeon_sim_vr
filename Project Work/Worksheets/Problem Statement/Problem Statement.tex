%%%%%%%%%%%%%%%%%%%%%%%%%%%%%%%%%%%%%%%%%
% Short Sectioned Assignment
% LaTeX Template
% Version 1.0 (5/5/12)
%
% This template has been downloaded from:
% http://www.LaTeXTemplates.com
%
% Original author:
% Frits Wenneker (http://www.howtotex.com)
%
% License:
% CC BY-NC-SA 3.0 (http://creativecommons.org/licenses/by-nc-sa/3.0/)
%
%%%%%%%%%%%%%%%%%%%%%%%%%%%%%%%%%%%%%%%%%

%----------------------------------------------------------------------------------------
%	PACKAGES AND OTHER DOCUMENT CONFIGURATIONS
%----------------------------------------------------------------------------------------

\documentclass[paper=a4, fontsize=11pt]{scrartcl} % A4 paper and 11pt font size

\usepackage[T1]{fontenc} % Use 8-bit encoding that has 256 glyphs
\usepackage{fourier} % Use the Adobe Utopia font for the document - comment this line to return to the LaTeX default
\usepackage[english]{babel} % English language/hyphenation
\usepackage{amsmath,amsfonts,amsthm} % Math packages

\usepackage{lipsum} % Used for inserting dummy 'Lorem ipsum' text into the template

\usepackage{sectsty} % Allows customizing section commands
\allsectionsfont{\centering \normalfont\scshape} % Make all sections centered, the default font and small caps

\usepackage{fancyhdr} % Custom headers and footers
\pagestyle{fancyplain} % Makes all pages in the document conform to the custom headers and footers
\fancyhead{} % No page header - if you want one, create it in the same way as the footers below
\fancyfoot[L]{} % Empty left footer
\fancyfoot[C]{} % Empty center footer
\fancyfoot[R]{\thepage} % Page numbering for right footer
\renewcommand{\headrulewidth}{0pt} % Remove header underlines
\renewcommand{\footrulewidth}{0pt} % Remove footer underlines
\setlength{\headheight}{13.6pt} % Customize the height of the header

\numberwithin{equation}{section} % Number equations within sections (i.e. 1.1, 1.2, 2.1, 2.2 instead of 1, 2, 3, 4)
\numberwithin{figure}{section} % Number figures within sections (i.e. 1.1, 1.2, 2.1, 2.2 instead of 1, 2, 3, 4)
\numberwithin{table}{section} % Number tables within sections (i.e. 1.1, 1.2, 2.1, 2.2 instead of 1, 2, 3, 4)

\setlength\parindent{0pt} % Removes all indentation from paragraphs - comment this line for an assignment with lots of text

%----------------------------------------------------------------------------------------
%	TITLE SECTION
%----------------------------------------------------------------------------------------

\newcommand{\horrule}[1]{\rule{\linewidth}{#1}} % Create horizontal rule command with 1 argument of height

\title{	
\normalfont \normalsize 
\textsc{Aalborg University, School of Information and Communication Technology} \\ [25pt] % Your university, school and/or department name(s)
\horrule{0.5pt} \\[0.4cm] % Thin top horizontal rule
\huge Problem Statement \\ % The assignment title
\horrule{2pt} \\[0.5cm] % Thick bottom horizontal rule
}

\date{\normalsize\today} % Today's date or a custom date

\begin{document}

\maketitle % Print the title

%----------------------------------------------------------------------------------------
%	PROBLEM 1
%----------------------------------------------------------------------------------------

\section{Motivation}

Robot assisted minimally invasive surgery (RAMIS) is used for many medical procedures. It allows for more precise surgery and lowers the risk of injury while reducing recovery time for patients \citep{radojcic_[history_2009}. Traditionally, the surgical performance has been measured in terms of surgeons' visuo-motor skills and dexterity, however a shift in the conceptualization of competencies is emerging in new literature \citep{flin_safer_2009}[p. 83]. One of these competencies is teamwork in the operating theatre (source Hayley et al). A study concluded that 80\% of incidents in anaesthesia related errors were preventable \citep{chopra_reported_1992}. Of these, 75\% of these were due to human error. 

Lingard et al studied communication in the operating theatre and found that 36\% of errors affected the teamwork negatively \citep{lingard_communication_2004}. From this study, five key points of communication in the operating theatre were found: time, resources, roles, safety and sterility, and situation control.


To counter human errors, and to reduce critical events, simulation based training has become increasingly popular over the years \citep{aggarwal_simulated_2004}. This applies to multiple professional and high risk contexts such as aviation and surgery. As the goal of simulated training is to involve all members of the surgical staff and to enhance their performance, R. Aggarwal suggests that the method can also be applied to other procedures such as robot surgery. As robot surgery is also a high risk context, virtual simulations of high risk procedures is also beneficial. 
This paper will explore the possibilities of simulating a medical emergency during surgery training in Virtual Reality (VR) and document the effects on teamwork and performance. 


%------------------------------------------------

\subsection{Problem Statement}

Based on the information above and focusing on the impact of recent Virtual Reality technology, this paper aims to answer the following problem statement:\\

\begin{center}
\emph{\textit{"How can virtual reality be utilized to simulate an emergency situation during RAMIS training?"}}
\end{center}


\subsection{Success Criteria}

Up for discussion. What do you think Martin? 

Suggestions: 

- have a working prototype
- have a way of measuring teamwork
- have a way of measuring performance
- have a procedure for testing the above and plan statistical test

%----------------------------------------------------------------------------------------

\end{document}