\section{Discussion}
The current system was made to test the viability of a fully implemented training simulation. The training sessions at MIUC focuses on certain key points such as communication, precision, and hands-on experience. However, implementing a system that is capable of simulating entire training scenarios where all these competencies are in focus is not possible with the available resources. As such, emergency undocking was chosen as it was deemed important for working with the da Vinci Surgical System as to not increase trauma risks for the patient.

The system was made for multiplayer use as communication and teamwork are key factors in successful surgeries. As observed in team training, it is important for trainees to know their tasks and roles of the rest of the medical personnel. Another important factor in the team training is to get familiar with the robot and increase ones time spent with it. As such, a single-player version of the simulation could be developed to allow for using the system without any other participants. Furthermore, having a single-player simulation would also allow the user to perform all the tasks required for surgery and thus learn the tasks of every member of the surgical team.

%In order to validate the usability of the system it is important to use participants already familiar with the HTC Vive. As learned from the user studies some users would learn how to use the VR system rather than learn the actual training from the simulation.