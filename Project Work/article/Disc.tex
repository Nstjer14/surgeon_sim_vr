\section{Discussion}
The current system functions as part of a could-be complex system that could include an entire simulation of a training session. The training sessions at MIUC focuses on certain key points such as communication, precision, and hands-on experience. However, implementing a system that is capable of simulating explicit scenarios where all these competencies will be in focus is not possible with the given time resources. As such, focusing on a certain task that is important for all working with the da Vinci Surgical System is chosen. Specifically, emergency undocking is required to be performed flawlessly as to not increase trauma risks for the patient.

The system is made solely for multiplayer usage as the communication and how to work as a team is mentioned as a key factor in successful surgeries. As observed in team training, learning to work with each other and what the tasks of the other participants are is important to know. However, as another important factor in the training is to get familiar with the robot and increase ones time spent with it. As such, a singleplayer version of the simulation could be developed to increase replayability as it would allow for using the system without other participants. Furthermore, having a singleplayer simulation would also allow the user to perform all the tasks required for surgery and thus learn the tasks of every member of the surgical team.

In order to validate the usability of the system it is important to use participants already familiar with the HTC Vive. As learned from the user studies some users would learn how to use the VR system rather than learn the actual training from the simulation.

XX dont know Olivers results XX