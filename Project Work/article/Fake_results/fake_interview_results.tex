\subsection{Expert Review Results}

Jane Petersson and Johan Poulsen were both able to test the system and observe footage from the usability test. The following interview revealed that the system at present was satisfactory at simulating the scenario, but was not comprehensive enough to warrant implementation with their students. They believed the interaction was sufficient, and that trainees didn't require additional features, however they both agreed that several features should be available to the instructors, such as scene change, interacting with procedures and progress, as well as changing the rules during play (for example by introducing emergencies). 

Jane Petersson thought that there was no need for the system to very accurately represent the actual scenario, but that approximation was sufficient. Docking and undocking was best trained in real life at a real system, and there was no need to attempt to replicate the physics in VR. She stated, however, that tactile feedback is important in real surgeries and training, but this can only be simulated using controller vibration in HTC Vive controllers.

Johan Poulsen agreed that there were limitations, but that systems such as this could be a must-have for the future of RAMIS. He thought enabling training around the world would advance both international RAMIS procedures, thus improving the outcome of individual surgeries, but also the training of specific scenarios, as the system would work well as an introduction and addition to real-life training at a licensed facility.