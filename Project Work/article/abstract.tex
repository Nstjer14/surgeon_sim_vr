The evolution of surgical robotics is increasing and continues to do so. Training and experience with the robots is a key factor to successful operations but is a costly event, as the equipment used is expensive and has a short lifetime.

To explore the possibility of enabling alternative robot assisted surgery training methods, we design and test a virtual reality simulation of emergency handling practised in certified institutes. This is done in cooperation with Minimal Invasive Education Centre and with Jane Petersson, First Nurse Assistant and Nurse specialist in Robot Surgery and Johan Poulsen, head surgeon at Aalborg University Hospital. To this end, a contextual study was conducted to ensure realism and accuracy of the procedure.

In order to asses the system's viability, we conduct both a usability test and expert review. The usability test is conducted with students from Aalborg University, and the expert review with both Jane Petersson and Johan Poulsen.

The results show a usability score of $67.5$ showing the usability to be average to good. The expert review was positive for the system's future, however it was considered too incomprehensive to consider implementation at this stage. More scenarios would be required in future implementations to allow for near full training sessions to be performed in VR as an addition to conventional training.