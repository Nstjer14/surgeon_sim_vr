The evolution of surgical robots is increasing and continues to do so. Training and experience with the robots is a key factor to successful operations but is a costly event, as the equipment used is expensive and has a short lifetime.

To ease training and decrease training cost, a virtual reality simulation is created. This is done in cooperation with the Minimal Invasive Education Centre and with Jane Petersson, First Nurse Assistant and Nurse specialist in Robot surgery and Johan Poulsen, head surgeon at Aalborg University Hospital.
This is done as an exploratory experiment to test if it is a viable solution to the problem at hand.

Viability checking is done by performing both a usability and utility test. The usability test is performed as a quantitative test where the utility test is an expert review with both Jane Petersson and Johan Poulsen. A contextual study has been carried out to gather the information needed to develop the system.

The results show a usability score of $67.5$ proving the usability to be ???
The utility test showed an acceptance of the proof-of-concept simulation made. Therefore the product developed is a success and proves viability in further development of the simulation.