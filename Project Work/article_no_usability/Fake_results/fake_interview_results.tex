\subsection{Expert Review Results}

Jane Petersson and Johan Poulsen were both able to test the system. The following interview revealed that the system at present was satisfactory at simulating the scenario, but was not comprehensive enough to warrant implementation with them. They believed the interaction was sufficient, and that trainees didn't require additional features, however they both stated that several features should be available to the instructors, such as scene change and reset, interacting with procedures and progress, as well as changing the rules during play (for example by introducing emergencies).

One key point throughout the interview was realism. This was brought up many times during the interview, and the consensus was that the closer the simulation was to total realism, both with controls, models, and textures, the more it could replace or improve upon current training standards. This meant that, for example, the robot should have multiple end effectors to allow more realistic movement. The current da Vinci model has three control points and can rotate around itself which the simulated model does not. In addition, the robot itself should be moveable. Introducing tool ports would, together with the improved handling, allow for teaching docking and undocking in VR. Despite all this, draping the robot arms would need to be practiced at real facilities due to the need for accurate tactile feedback.

Johan Poulsen stated that there were limitations, but that systems such as this could be a must-have for the future of RAMIS. He talked about fully integrating the system with current simulators such as RAMIS console and anaesthetic nurse simulators. This could allow full surgery team training. As noted from the context study, Jane Petersson sabotages the robot setup during team training in order to train the nurses' and doctors' communication and emergency handling skills. Being able to control even more factors in virtual reality, such as patient fever, would allow for more detailed training of both nurses and surgeons. This concept could also be extended to include scenarios and scenario control. By being able to reset the scene and load a scenario where the setup is an appendectomy would also improve the utility of the system thereby having multiple scenarios to choose from. 

Despite its limitations, the current version could be used to introduce medical students to RAMIS if realism was improved. Johan Poulsen also mentioned that, as there was no tactile feedback, the visuals in the scene had to become more realistic as the vision would overcompensate for the lack of tactile feedback. 

Despite the current limitations both experts agreed that the controls of the system were intuitive and that the learning curve was appropriate for their level of expertise with VR. Observations also showed that they both learned to teleport, grab, and interact with the robot fast. 


%Jane Petersson thought that there was no need for the system to very accurately represent the actual scenario, but that approximation was sufficient. Docking and undocking was best trained in real life at a real system, and there was no need to attempt to replicate the physics in VR. She stated, however, that tactile feedback is important in real surgeries and training, but this can only be simulated using controller vibration in HTC Vive controllers.